% Footer Formatting: DO NOT CHANGE -----------------------------------
% Setting the page numbers to be arabic 
\pagenumbering{arabic}

\fancyfoot{}

\fancyfoot[LE]{\colorbox{urban-footergray}{\makebox(0.2, 0.12)[r]{\fontsize{7.5}{0}\selectfont\bfseries{\MakeUppercase{}\hspace{0.2in}}}}\colorbox{urban-gold}{\makebox(0.2, 0.12)[c]{\fontsize{7.5}{0}\selectfont\bfseries{\MakeUppercase\thepage}}}\colorbox{urban-footergray}{\makebox(5.64, 0.12)[r]{\fontsize{7.5}{0}\selectfont\bfseries{\MakeUppercase{\so{\THETITLE}}\hspace{0.2in}}}}}

\fancyfoot[RO]{\colorbox{urban-footergray}{\makebox(5.64, 0.12)[l]{\fontsize{7.5}{0}\selectfont\bfseries{\hspace{0.2in}\MakeUppercase{\so{\THETITLE}}}}}\colorbox{urban-gold}{\makebox(0.2, 0.12)[c]{\fontsize{7.5}{0}\selectfont\bfseries{\MakeUppercase\thepage}}}\colorbox{urban-footergray}{\makebox(0.2, 0.12)[r]{\fontsize{7.5}{0}\selectfont\bfseries{\MakeUppercase{}\hspace{0.2in}}}}}
% Footer Formatting: DO NOT CHANGE -----------------------------------

%%%%%%%%%%%%%%%%%%%%%%%%%%%%%%%%%%%% Report %%%%%%%%%%%%%%%%%%%%%%%%%%%%%%%%%%%%%

% Report headers examples -----------------------------------
\part{Report Title Here in Title Case (Heading 1 style)}

Body Text style for first paragraph.

Body Text First Indent style for subsequent paragraphs. \textcolor{red}{To add an endnote, use the Insert Endnote function. Endnotes will automatically appear in the Notes section after the appendixes.}\endnote{Endnotes inserted in the executive summary, main report, and appendixes will all display here if section breaks are used correctly.} 

\section{First-Level Heading in Title Case (Heading 2 style)}

Body Text style for first paragraph under a heading.

Body Text First Indent style for all subsequent paragraphs.
% Bullet examples -----------------------------------
\begin{itemize}
    \item Bulleted List style for bulleted lists.
    \item Another bullet using Bulleted List style.
    \begin{itemize}
        \item Bulleted List 2 style for second-level bulleted items under a numbered or bulleted list item.
        \item Another item using Bulleted List 2 style.
    \end{itemize}
    \item Another bullet using Bulleted List style.\\
    Indented Text style for ordinary paragraphs under a numbered or bulleted list item.
\end{itemize}

\noindent Body Text First Indent style for all subsequent paragraphs.

\begin{enumerate}
    \item Numbered List style for numbered lists
    \begin{enumerate}
        \item Numbered List 2 style for second-level lettered items under a numbered or bulleted list item.
        \item Another paragraph using Numbered List 2 style.
    \end{enumerate}
    \item Another item using Numbered List style.\\
    Indented Text style for ordinary paragraphs under a numbered or bulleted list item.
\end{enumerate}

\noindent Body Text First Indent style for all subsequent paragraphs.

\subsection{Second-Level Heading in Title Case (Heading 3 style)}

\indent Body Text style for first paragraph under a heading.

\subsubsection{Third-level heading; all caps is built into style (heading 4 style)}

Body Text style for first paragraph under a heading.

\fourthsub{Fourth-level headings are run into the text.} Select the first clause or sentence of your paragraph, then apply the Heading 5 (D) style. 

% Quote examples -----------------------------------
% TODO: Still need to code quotes up
Pull Quote style for pull quotes.

---Pull quote attribution in sentence case

Body Text First Indent style for all subsequent paragraphs.

\block{Block Text style for longer (100+ words) quotes and excerpts.}

% Figures and Tables examples -----------------------------------
\newpage
\part{Figures and Tables}

\begin{figure}[htbp]
    \caption{(Figure/Table Number Style)}
    \centering
    \includegraphics[width=0.5\textwidth]{images/logo.png}
    \label{fig:my_label}
\end{figure}
\begin{singlespace}
    \source{\textbf{Source:} Figure/Table Notes style. Source in sentence case, except proper nouns. Change ``Source:'' to bold.}\\
    \source{\textbf{Notes:} Figure/Table Notes style. Notes in sentences (and sentence case). Change ``Notes:'' to bold.}
\end{singlespace}

% Tables get tricky in LaTeX, because you have to specify each row and column.
% These two examples mirror the Urban Report Word Doc.
% Some ways to make LaTeX table without much work:
% 1. Use xtable in R to generate the LaTeX code for a table you generated in R.
% 2. https://www.tablesgenerator.com/, a LaTeX table generator

\begin{singlespace}
    \begin{table}[htbp]
        \def\arraystretch{1.2}
        \caption{(Figure/Table Number Style)}
        \label{tab:my_label}
        \centering
        \begin{tabular}{L{0.2\textwidth} C{0.2\textwidth} C{0.2\textwidth} C{0.2\textwidth} C{0.2\textwidth}}
                & \multicolumn{2}{C{0.4\textwidth}}{\tabheader{Spanner Heading in Title Case (Table Column Heading Style)}} & \multicolumn{2}{C{0.4\textwidth}}{\tabheader{Spanner Heading in Title Case (Table Column Heading Style)}}\\
            \hline
                & \tabheader{Column labels in sentence case (Table Column Heading style)} & \tabheader{Column labels in sentence case (Table Column Heading style)} & \tabheader{Column labels in sentence case (Table Column Heading style)} &  \tabheader{Column labels in sentence case (Table Column Heading style)}\\
            \hline
                Table row heading in sentence case (Table Row Heading style) &  &  &  &  \\ 
                Table subheading in sentence case (Table Row Subheading style) &  &  &  &  \\ 
                Table row in sentence case (Table Row style) &  &  &  &  \\
            \hline
        \end{tabular}
    \end{table}
    \source{\textbf{Source:} Figure/Table Notes style. Source in sentence case, except proper nouns. Change ``Source:'' to bold.}
    \source{\textbf{Notes:} Figure/Table Notes style. Notes in sentences (and sentence case). Change ``Notes:'' to bold.}
\end{singlespace}

% Reference examples -----------------------------------
\cleardoublepage
\part{Reference Styles}
See them printed out in the References section.

\section{Reference Styles for Urban Products}

\subsection{Blog posts (cite in notes, not reference list)}

\subsection{Policy debates (cite in notes, not reference list)}

\textbf{The entire debate:}

\noindent\textbf{One post in the debate:}

\section{Books}

\begin{itemize}
    \item \textbf{Single Author Book:} \citet{schwabish2020elevate}
    \item \textbf{Multiple Authors:} \citet{burman2020taxes}
    \item \textbf{Editor(s) in place of author:}
    \item \textbf{Chapter in an edited book:}
    \item \textbf{Urban Institute Press book:} \citet{cordes2009nonprofits}
    \item \textbf{One volume in multi-volume work:} \citet{bowen2021book}
    \item \textbf{Numbered edition:}
\end{itemize}

